\section{Estado del Arte}
%TEXTO
\begin{frame}{Clasificaci\'on con modelos de texto}

   \begin{itemize}
    \item \textit{Transformer encoder} preentrenado: \textit{Robustly optimized BERT approach} (RoBERTa) \footfullcite{liu2019robertarobustlyoptimizedbert}
    \item \textit{Fine-tuning} junto con \textit{token} especial para la  clasificaci\'on ([CLS])
    \item Modelo monoling\"ue preferible a uno multiling\"ue
    \item Tareas destacables: EXIST, DETESTS, AMI (\textit{Automatic Misogyny Identification})
\end{itemize}
\end{frame}

%MULTIMODAL
\begin{frame}{Clasificaci\'on con modelos multimodales}
    \begin{itemize}
        \item \textit{Early Fusion}: Fusionar modalides de arquitecturas unimodales a trav\'es de alg\'un mecansimo
        \item Modelos preentrenados orientados a alinear texto e imagen:
        \begin{itemize}
            \item CLIP (\textit{Contrastive Language-Image Pre-Training})
            \item VisualBERT
            \item FLAVA (\textit{Foundation Language And Vision Alignment})
        \end{itemize}
        \item Texto $\gg$ Imagen  para los memes 
        \item Tareas destacables: MAMI (Multimedia Automatic Misogyny Identification), Hateful Memes Challenge
    \end{itemize}
\footfullcite{aggarwal-etal-2024-text}
\end{frame}

%LeWiDi
\begin{frame}{LeWiDi}
\begin{itemize}
    \item Enfoque cl\'asico: 
    \begin{itemize}
        \item Asumir la existencia de una verdad absoluta
        \item Votaci\'on mayoritaria o m\'etodos probabil\'isticos
        \item Mejor para la evaluaci\'on \textit{hard}
    \end{itemize}
    \item Enfoque perspectivista:
    \begin{itemize}
        \item Modelar las anotaciones individuales
        \item \textit{Ensemble}, \textit{multi-label} y \textbf{\textit{multi-task}}
    \end{itemize}
    \item Enfoque \textit{soft loss}: 
    \begin{itemize}
        \item Agregar las anotaciones en una distribuci\'on: \textit{Softmax} o emp\'irica 
        \item $\mathcal{L}$: \textbf{\textit{Cross Entropy}}, Kullback-Leibler y \textit{Minimum Square Error}
        \item Enfoque que m\'as generaliza
    \end{itemize}
\end{itemize}
\footfullcite{lewidi_survey}
\end{frame}

\begin{frame}{M\'etricas}
    \begin{columns}[T]
        \column{0.5\textwidth}
        \centering
        \textbf{Evaluaci\'on \textit{hard}}
        \vspace{0.25cm}
        \begin{itemize}
            \item $F_1$ \textit{score}:
            $$ F_1 =  2\frac{P \cdot R}{P + R} $$
            \item \textit{Information Contrast Metric} (ICM) 
            \begin{align*}
                ICM(A, B) &= \alpha_1IC(A) + \alpha_2IC(B) \\
                          &- \beta IC(A \cup B) \\
                IC(A) &= -\log_2P(A)
            \end{align*}
        \end{itemize}
        
        \column{0.5\textwidth}
        \centering
        \textbf{Evaluaci\'on \textit{soft}}
        \vspace{0.25cm}
        \begin{itemize}
            \item \textit{Cross Entropy}
            \begin{align*}
                CE(p, q) = - \sum_{x \in \mathcal{X}} p(x) \log_2q(x)
            \end{align*}
            \item \textit{ICM Soft}: ICM adaptada para distribuciones de probabilidad
            $$
            IC(\{c_i, v_i\}) = -log_2(P(\mathcal{I}_c \geq v))
            $$
        \end{itemize}
    
    \end{columns}
    \footfullcite{amigo-delgado-2022-evaluating}
\end{frame}